\documentclass[12pt]{article}
\usepackage[top=1in, left=1in, bottom=1in, right=1in]{geometry}
\usepackage{preamble/preamble}
\usepackage[sort&compress]{natbib}

\author{J. F. Dooley}
\title{\LaTeX{} template using \texttt{latexmk}}

\begin{document}

\maketitle

\section{Hello World!}
This is a minimal example of the latex preamble and build workflow used during graduate school.
This template and custom preamble are both available to the public at \href{https://github.com/jfdoolster}{github.com}.

\section{Shortcuts}
The main purpose of this template is for my personal use when building a new latex document. The preamble provides a variety of typesetting shortcuts for units, scientific notation, and Latin phrases.
%
Units: $1\Units{kg} = 1000\Units{g} = 1\E{9}\Units{\Micro{}m} = 1\e{9}\Units{\Micro{}m}$.
%
Compounds: \Methane{}, \CarbonDioxide{}, \& \Ethane{} are hydrocarbons produced during Oil and Natural Gas (\ONG{}) extraction.

\section{Coursework Prompts}

\Prompt{Question: What is \Degrees{270} in radians? What is $1.75\pi\Units{radians}$ in degrees?}
%
A unit circle has \Degrees{360} or $2\pi\Units{rad}$, therefore $\Degrees{270} \cdot \left(\pi/180\right) = \boxed{3\pi/4 \Units{rad} = 2.356\Units{rad}}$. Similarly, $1.75\pi\Units{rad} \cdot \left(180/\pi\right) = \boxed{\Degrees{315} = \Degrees{-45}}$.


\section{Code Blocks}
Blocks of code can be inserted using the \href{hhttps://ctan.org/pkg/listings}{listings} package.

\begin{lstlisting}[language=Python, caption={Python Code Example}]
if __name__=="__main__":
    print("hello world!")
\end{lstlisting}

\lstinputlisting[language=Python, caption={Python code from file}]{./main.py}



\section{Filler Text}
Sometimes it is helpful to add filler text to your document to make sure the layout is correct. The \href{https://ctan.org/pkg/lipsum}{lipsum} package easily accomplishes this task~\cite{LaTeX2020,Kerrisk2010}.
\lipsum[1]


\bibliographystyle{unsrt}
\bibliography{refs}

\end{document}